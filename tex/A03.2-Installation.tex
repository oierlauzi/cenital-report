\documentclass[../main.tex]{subfiles}

\begin{document}

This software can be installed in two ways: Release and development. The first manner is advised for normal use. The second one should only be used when adding new features to the source code.\newline

\subsection{Release installation}

The first step is to install all the dependencies. On Debian based systems this can be done using the following command:

\begin{lstlisting}[language=bash]
    sudo apt install\
    libvulkan1 vulkan-validationlayers\
    libglfw3\
    libavutil56 libavformat58 libavcodec58 libswscale5
\end{lstlisting}

Copy all the shared libraries to the \texttt{/usr/local/lib/} directory and update registries:

\begin{lstlisting}[language=bash]
    sudo cp /path/to/provided/files/lib/* /usr/local/lib/
    sudo ldconfig
\end{lstlisting}

Copy the executable to the \texttt{/usr/local/bin/} directory:

\begin{lstlisting}[language=bash]
    sudo cp /path/to/provided/files/bin/cenital /usr/local/bin/
\end{lstlisting}


\subsection{Development installation}

Firstly, install all the development tools required to build the binaries. In this example, the GNU C compiler (GCC) is used, but the reader may choose an alternative compiler such as Clang.\newline

\begin{lstlisting}[language=bash]
    sudo apt install g++ cmake
\end{lstlisting}

Then install all the dependencies using the following command:

\begin{lstlisting}[language=bash]
    sudo apt install\
    libvulkan-dev vulkan-validationlayers-dev glslang-dev\
    libglfw3-dev\
    libavutil-dev libavformat-dev libavcodec-dev libswscale-dev
\end{lstlisting}

Now proceed to build and install all the \textit{Zuazo} modules. This can be automated using the following script:

\begin{lstlisting}[language=bash]
    cd /path/to/provided/files/src
    for d in zuazo*/; do
        cd $d
        mkdir build
        cd build
        cmake ../
        make -j16
        sudo make install
        cd ../..
    done
\end{lstlisting}

When a line of code inside one of these libraries is changed, it needs to be rebuilt using the following commands. Moreover, if a file is added or deleted, \texttt{cmake ../} must be executed after \texttt{cd}.\newline

\begin{lstlisting}[language=bash]
    cd /path/to/provided/files/src/module/build
    make -j16
    sudo make install
\end{lstlisting}

Finally, build \textit{Cenital} itself:

\begin{lstlisting}[language=bash]
    cd /path/to/provided/files/src/cenital
    mkdir build
    cd build
    cmake ../
    make -j16
    sudo make install
\end{lstlisting}


\end{document}
