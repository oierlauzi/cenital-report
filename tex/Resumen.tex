\documentclass[../main.tex]{subfiles}

\begin{document}
Cualquier producción audiovisual en vivo requiere de un mezclador de vídeo para poder componer y conmutar entre fuentes de vídeo. Pese a que su invención se remonta a los comienzos de las retransmisiones televisivas, siguen siendo un sistema clave en todos los programas en directo que empleen más de una fuente de vídeo (la inmensa mayoría).\newline

Se puede asegurar que siempre habrá programas que, dada su naturaleza, el público quiera disfrutar de ellas en directo, como es el caso de algunos eventos deportivos, informativos o culturales. Pese a que en la última década el panorama audiovisual ha sufrido un vuelco con la disrupción del streaming, siempre que se sigan usando sistemas de vídeo para dar cobertura a estos eventos, los mezcladores seguirán existiendo de una u otra forma.\newline

La larga trayectoria de los mezcladores no implica que no exista lugar para la innovación. Es más, dado que se encuentran tan estrechamente relacionados con un campo en rápido desarrollo como es la multimedia, existe mucho interés en mejorarlos. Una de las innovaciones que se están llevando a cabo en el ámbito de los mezcladores de vídeo es la utilización de software para emular estos sistemas. Esto presenta varias ventajas, primordialmente relacionadas con la flexibilidad, la agilidad y el bajo costo de la solución adoptada.\newline  

Por otro lado, también se está ahondando en la posibilidad de usar redes de ordenadores para el transporte de señales de vídeo dentro de un estudio de televisión, lo que se conoce como producción IP. Esto presenta ventajas muy similares a las descritas para los mezcladores definidos por software y además estas dos tecnologías se compenetran excelentemente.\newline

El presente proyecto consiste en la implementación de un mezclador definido por software que adquiera las señales de vídeo a través de la red. Tiene como objetivo perseguir las ventajas que ofrecen estas dos tecnologías. Además, pretende realizar nuevas aportaciones al campo de los mezcladores permitiendo recortar cuadros de vídeo por curvas paramétricas.\newline

\end{document}
