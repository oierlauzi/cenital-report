\documentclass[../main.tex]{subfiles}

\begin{document}
Vision mixers are used in all live broadcasts to switch and composite among all available video feeds. Despite them being introduced in the early days of television, they are still a key system in all live productions that use more than one video source (almost all).\newline

The audience will always demand live coverage of some events due to their nature, as is the case of sporting events, news and cultural happenings. In the last decade, the disruption of video streaming has revolutionized the multimedia industry. In spite of this, vision mixers continue to exist in one form or another, regardless of the used transmission medium.\newline

Even though vision mixers are a very old technology, this does not imply that there is no room for innovation in this field. What is more, as they are tightly related to a rapidly changing industry, as is the case of the multimedia sector, there is a lot of interest on improving them. One of the most important ongoing advancements in the field of vision mixers is the usage of computer software to emulate their functionality. This has several advantages, such as increased flexibility, agility and lower operation costs.\newline

Additionally, there is a trend of using computer networks to carry video signals across television production facilities. This is known as IP production and it offers similar advantages to the ones enumerated earlier. Moreover, this technology fits very well with the software-based vision mixers.\newline

This project consists in implementing a software-based vision mixer that ingests video from the network. The project pursues all the benefits of these two technologies. In addition, it pretends to introduce some features that are not offered by commercial mixers. For instance, it is able to crop video frames with an arbitrary shape defined by parametric curves.\newline

\end{document}
