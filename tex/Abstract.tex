\documentclass[../main.tex]{subfiles}

\begin{document}
The vision mixer is used in all live broadcasts to switch and composite among all available feeds. Despite being introduced in the early days of television broadcasts, it is still a key system in all multi-camera productions.\newline

The audience will always demand live coverage of some events due to their nature, as is the case of sporting events, news and cultural happenings. Therefore, in spite of the disruption of video streaming technologies, vision mixers continue to exist in one form or another regardless of the transmission medium used.\newline

Even though vision mixers are a very old technology, this does not imply that there is no innovation in this field. What is more, as they are tightly related to the multimedia sector, there is a lot of interest to improve them. One of the most important ongoing advancements in the field of vision mixers is the usage of computer software to emulate their functionality. This has several advantages, such as increased flexibility, agility and lower operation costs.\newline

Additionally, there is a trend of using computer networks to carry video signals across a television production facility. This is called as IP production and offers similar advantages to the ones enumerated earlier. Moreover, this technology fits very well with the software-based vision mixers.\newline

This project consists in implementing a software-based vision mixer that ingests video from the network. The project pursues to achieve all the benefits of these technologies. In addition, it pretends to expand on the features offered by commercial mixers. To do so, it implements the ability to crop video frames with an arbitrary shape defined by parametric curves.\newline

\end{document}
